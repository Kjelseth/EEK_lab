\documentclass[conference]{IEEEtran}

% --- Common packages ---
\usepackage{graphicx}    % figures
\usepackage{amsmath}     % equations
\usepackage{cite}        % IEEE-style citations
\usepackage{hyperref}    % clickable references
\usepackage{array}       % tables

\usepackage{siunitx}
\sisetup{
  detect-all,       % matches the surrounding font
  separate-uncertainty = true,
  per-mode = symbol  % e.g., m/s instead of m s^-1
}


\begin{document}

% --- Title & Authors ---
\title{Lab Report: \LaTeX{} Test with IEEEtran}

\author{%
    John Doe \\
    Department of Electrical Engineering, Example University \\
    Email: john.doe@example.com
}

\maketitle

% --- Abstract ---
\begin{abstract}
This is a sample abstract for a lab report using the IEEEtran class
with modern fonts. You can type Unicode symbols like Ω, µ, or α
without warnings.
\end{abstract}

% --- Keywords ---
\begin{IEEEkeywords}
lab, experiment, IEEEtran, LuaLaTeX, fonts
\end{IEEEkeywords}

% --- Sections ---
\section{Introduction}
This template is based on the official IEEEtran class but updated
to use \texttt{newtxtext} and \texttt{newtxmath} for full font support.
The measured voltage was \SI{5.0}{\volt} and the current was \SI{2.3}{\milli\ampere}.

The resistor has a value of \SI{10}{\kilo\ohm}.

The frequency of the signal is \SI{1.2e3}{\hertz}.

The length of the sample is \SI{12.3(1)}{\centi\meter} % 12.3 ± 0.1 cm

The acceleration is \SI{9.81}{\meter\per\second\squared}.

\begin{equation}
V = I \, R
\end{equation}

\begin{equation}
P = \frac{V^2}{R} = \frac{\SI{5}{\volt}^2}{\SI{10}{\ohm}} = \SI{2.5}{\watt}
\end{equation}



\section{Materials and Methods}
Describe apparatus, measurement setup, and procedure here.

\section{Results}
Insert figures and tables as needed.

\section{Discussion}
Interpret results and analyze sources of error here.

\section{Conclusion}
Summarize findings and lessons learned here.

% --- References ---
\bibliographystyle{IEEEtran}
\bibliography{references} % references.bib file

\end{document}
