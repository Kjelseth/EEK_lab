\documentclass{article}

%================  CUSTOMIZATION  =================================================%

\usepackage{KjelsethReportStyle_old}

%================  METADATA  ======================================================%

\title{\fontsize{24}{36}\selectfont Elektroniske enheter og kretser\\ % Input title
Lab 01} % Line 2 of title, \\ means next line.

\author{{\ttfamily Sølve Kjelseth}} % Input your name.
% replace \ttfamily with \normalfont to make it regular font.
% (removing \ttfamily will not do this automatically.

\date{\today} % Auto updates the date, until you export it.


%================  START OF DOCUMENT  =============================================%
\begin{document}

\maketitle % Makes title front page based on the title, author and date metadata, change at the top

%================  INTRODUCTION  ==================================================%
% #region
\addtocontents{toc}{\protect\setcounter{tocdepth}{0}} % Temporarily hide from TOC
\section{Introduction} % Numbered section named Introduction
This is the first report in this course, detailing the completion of the first lab exercise.\par
\vfill
Note: As always, the \LaTeX\ file and all other assets, such as text, images, graphs and code made by me for this project is open source with the MIT licence, see
\linkgithub[true][0.5]{my GitHub}

\clearpage

\tableofcontents % Generate TOC
\hfill
\listoffigures % List of figures
\hfill
\listoftables % List of tables
\addtocontents{toc}{\protect\setcounter{tocdepth}{2}} % Restore TOC depth

% #endregion

%================  SECTION  =======================================================%
\section{Part 1 - Diode test}
% #region
This Part is about testing a diode characteristics with a multimeter. This means it is inherently not perfect, but it will function as a reference measurement.

%================  SINGLE FIGURE  =================================================%
\begin{figure}[h] % Circuit
    \centering
    \includegraphics[width=\textwidth]{Part1.jpg}
    \caption{Diode being measured}
    \label{fig:part1}
\end{figure}

%================  TABLE  =========================================================%
% Table generated by Excel2LaTeX from sheet 'Sheet1'
\begin{table}[htbp] % Measurements
  \centering
  \caption{Diode measurements}
    \begin{tabular}{|l|lr|l|lr|}
    \hline
    Voltage forward & \multicolumn{1}{r}{0.593} & \multicolumn{1}{l|}{V} & Resistance forward & \multicolumn{1}{r}{225400} & \multicolumn{1}{l|}{\Omega} \bigstrut\\
    \hline
    Voltage reverse & OL    &       & Resistance reverse & OL    &  \bigstrut\\
    \hline
    \end{tabular}%
  \label{tab:part1}%
\end{table}%


Interesting to note that the measured resistance in forward-bias of the diode fluctuated a lot. It went into high \(\SI{}{\mega\ohm}\) to low tens of \(\SI{}{\kilo\ohm}\). It was most stable around \(\SI{200}{\kilo\ohm}\) and one of these measurements was therefore noted down. This could be because the multimeter is acting as a power supply in resistance measuring mode and depending on the voltage chosen by the auto ranging multimeter the diode behaviour differs.

% #endregion

%================  SECTION  =======================================================%
\section{Part 2 - Forward-bias characteristics}
% #region
This Part is about testing the diode characteristics for forward-bias. The values was stored in a table (RAW data like this is found on the \linkgithub{GitHub}) and then a plot was made to compare the current through the diode \(I_D\) with the voltage drop over the diode \(V_D\).

%================  SINGLE FIGURE  =================================================%
\begin{figure}[h] % Circuit
    \centering
    \includegraphics[width=\textwidth]{Part2.jpg}
    \caption{Forward-bias circuit}
    \label{fig:Part2}
\end{figure}

\clearpage

%================  SINGLE FIGURE  =================================================%
\begin{figure}[h] % Plot characteristics
    \centering
    \includegraphics[width=\textwidth]{Part2Plot.png}
    \caption{Plot of forward-bias characteristics}
    \label{fig:Part2Plot}
\end{figure}

Now when when extending the plot all the way to the origin it gets a characteristics that looks a lot different. As seen in Figure~\ref{fig:Part2PlotExtended} it looks like after the initial curve the value gets linear.

\clearpage

%================  SINGLE FIGURE  =================================================%
\begin{figure}[h] % Plot extended
    \centering
    \includegraphics[width=\textwidth]{Part2PlotExtended.png}
    \caption{Extended plot of forward-bias characteristics}
    \label{fig:Part2PlotExtended}
\end{figure}

This looks now very much like an exponential relationship, when checked, the end result is very similar to to this function. This shows that the behaviour can be very accurately modelled easily using the exponential function.

\clearpage

%================  SINGLE FIGURE  =================================================%
\begin{figure}[h] % Plot regression
    \centering
    \includegraphics[width=\textwidth]{Part2PlotExp.png}
    \caption{Extended plot with exponential regression}
    \label{fig:Part2PlotExp}
\end{figure}

% #endregion

%================  SECTION  =======================================================%
\section{Part 3 - Reverse-bias}
% #region
This Part is about testing the reverse-bias current. Measurements was made and noted in the table, note that the assumed resistive value of the voltmeter is specified by the assignment.

%================  SINGLE FIGURE  =================================================%
\begin{figure}[h] % Circuit
    \centering
    \includegraphics[width=\textwidth]{Part3.jpg}
    \caption{Reverse-bias circuit}
    \label{fig:Part3}
\end{figure}

\clearpage

%================  TABLE  =========================================================%
% Table generated by Excel2LaTeX from sheet 'Sheet1'
\begin{table}[htbp] % Measurements
  \centering
  \caption{Reverse-bias measurements}
    \begin{tabular}{|l|rl|}
    \hline
    \(E\) (Measured) & 20.03 & V \bigstrut\\
    \hline
    \(R_M\) (Assumed) & 10    & M\Omega \bigstrut\\
    \hline
    \(R\) (Measured) & 1002.5 & k\Omega \bigstrut\\
    \hline
    \(V_R\) (Measured) & 6.2   & mV \bigstrut\\
    \hline
    \(I_S\) (Calculated) & 6.805 & nA \bigstrut\\
    \hline
    \(R_{DC}\) (Calculated) & 2942.71 & M\Omega \bigstrut\\
    \hline
    \end{tabular}%
  \label{tab:Part3}%
\end{table}%

It looks as if the values for \(I_S\) and \(R_{DC}\) miss by an order of magnitude as the calculated reverse-bias resistance often leads to values between hundreds of \(\SI{}{\kilo\ohm}\) and up to a few hundred \(\SI{}{\mega\ohm}\). The inherent inaccuracies in the measurements are probably the cause of this error, if this actually is an error.

% #endregion

%================  SECTION  =======================================================%
\section{Part 4 - LED characteristics}
% #region
This part is about testing the characteristics of LED's. 

%================  SINGLE FIGURE  =================================================%
\begin{figure}[h] % Circuit
    \centering
    \includegraphics[width=\textwidth]{Part4.jpg}
    \caption{LED circuit}
    \label{fig:Part4}
\end{figure}

First the circuit was connected and then the voltage supply was slowly ramped up. when first light appeared the value was recorded. Although hard to see, in Figure~\ref{fig:FirstLight} there is a very faint sub-lumen light emitting from the diode. When the supply was ramped up until brightness levelled out at a bright level, the values were recorded again.

\clearpage

%================  TABLE  =========================================================%
% Table generated by Excel2LaTeX from sheet 'Sheet1'
\begin{table}[htbp] % Measurements
  \centering
  \caption{LED measurements}
    \begin{tabular}{|l|rl|rl|}
    \hline
    measurements & \multicolumn{2}{c|}{First light} & \multicolumn{2}{c|}{Bright} \bigstrut\\
    \hline
    \(V_D\) (Measured)   & 1.787   & V   & 2.185  & V \bigstrut\\
    \hline
    \(V_R\) (Measured)   & 17.8    & mV  & 3.632  & V \bigstrut\\
    \hline
    \(I_D\) (Calculated) & 179.980 & μA  & 36.724 & mA \bigstrut\\
    \hline
    \end{tabular}%
  \label{tab:part4}%
\end{table}%

\vspace{2em}

%================  MULTI-FIGURE (WITH SUBFIGURES)  ================================%
\begin{figure}[h] % LED light levels
    \centering
    \begin{subfigure}[t]{0.49\textwidth}
        \centering
        \includegraphics[width=0.9\textwidth]{Part4_FirstLight.jpg}
        \subcaption{First light}
        \label{fig:FirstLight}
    \end{subfigure}
    \hfill
    \begin{subfigure}[t]{0.49\textwidth}
        \centering
        \includegraphics[width=0.9\textwidth]{Part4_FullyOn.jpg}
        \subcaption{Bright}
        \label{fig:Bright}
    \end{subfigure}
    \caption{LED light levels}
    \label{fig:Part4States}
\end{figure}

Then multiple data points was collected for different input voltages, and the results are listed below in Table~\ref{tab:LED} and then graphed to display the data in varying ways.

%================  TABLE  =========================================================%
% Table generated by Excel2LaTeX from sheet 'Sheet1'
\begin{table}[htbp] % Full measurements
  \centering
  \caption{LED values}
  \resizebox{\textwidth}{!}{%
    \begin{tabular}{|l|r|r|r|r|r|r|r|}
      \hline
      \(E\) (V) & 0.000 & 1.033 & 2.008 & 3.008 & 4.002 & 5.010 & 6.040 \bigstrut\\
      \hline
      \(V_D\) (V) & 0.000 & 1.032 & 1.860 & 1.987 & 2.066 & 2.135 & 2.201 \bigstrut\\
      \hline
      \(V_R\) (V) & 0.000 & 0.000 & 0.146 & 1.019 & 1.934 & 2.875 & 3.840 \bigstrut\\
      \hline
      \(I_D\) (mA) & 0.000 & 0.000 & 1.480 & 10.303 & 19.555 & 29.070 & 38.827 \bigstrut\\
      \hline
    \end{tabular}%
  }
  \label{tab:LED}%
\end{table}

\clearpage

%================  SINGLE FIGURE  =================================================%
\begin{figure}[h] % Plot voltage drops
\centering
\includegraphics[width=\textwidth]{Part4PlotVoltageDrops.png}
\caption{Voltage drops}
\label{fig:part4voltage}
\end{figure}
Interesting to note that the voltages behave very linear after the initial two data points where the diode is not yet in its active region. The \(R^2\) value shows how accurately the approximation is and here it is very high.
\clearpage

%================  SINGLE FIGURE  =================================================%
\begin{figure}[h] % Plot voltage percent
\centering
\includegraphics[width=\textwidth]{Part4PlotPercent.png}
\caption{Plot of voltage drops as percent of source}
\label{fig:part4percent}
\end{figure}
Graphing the voltages as precents of the source is a good way of visualizing how voltages will continue to behave when further increasing voltage, assuming no breakdown.
\clearpage

%================  SINGLE FIGURE  =================================================%
\begin{figure}[h] % Plot stacked bars voltage percent
\centering
\includegraphics[width=\textwidth]{Part4PlotStacked.png}
\caption{Stacked bar chart of voltage drops as percent of source}
\label{fig:part4stacked}
\end{figure}
Using a stacked bar char shows this even more clear, and tells a story about how voltage is divided between the resistor and the diode.
\clearpage

%================  SINGLE FIGURE  =================================================%
\begin{figure}[h] % Plot current
\centering
\includegraphics[width=\textwidth]{Part4PlotCurrent.png}
\caption{Current through the diode vs voltage}
\label{fig:part4current}
\end{figure}
As the voltages increase linear the current should do as well, and here again, if the two first datapoint are excluded the the linearity is very good.

% #endregion

%================  SECTION  =======================================================%
\section{Part 5 - Zener characteristics}
% #region
This part is about a zener diode, which is a type of diode designed to work in reverse-bias in the breakdown region and not physically break, but continue to operate here.

%================  SINGLE FIGURE  =================================================%
\begin{figure}[h] % Circuit
\centering
\includegraphics[width=\textwidth]{Part5.jpg}
\caption{Zener diode circuit}
\label{fig:part5}
\end{figure}

The data table for the zener circuit is just all zeros but the last measurements, so instead of a table I want to represent it in graph form.

\clearpage

%================  SINGLE FIGURE  =================================================%
\begin{figure}[h] % Plot voltages
\centering
\includegraphics[width=\textwidth]{Part5PlotVoltageDrops.png}
\caption{Voltage drops}
\label{fig:part5voltage}
\end{figure}

After reaching the zener voltage, the rated voltage of the diode, the voltage drop over the diode is almost flat, you can see this is linear for this small dataset with a very low slope.

\clearpage

%================  SINGLE FIGURE  =================================================%
\begin{figure}[h] % Plot current
\centering
\includegraphics[width=\textwidth]{Part5PlotCurrent.png}
\caption{Current through the diode vs voltage}
\label{fig:part5current}
\end{figure}

As the voltage is almost flat over the diode, this means that the current through the diode will rise almost vertically, an this is shown clearly on the graph, as the slope is very high.

% #endregion

%================  SECTION  =======================================================%
\section{Part 6 - Half-wave rectification}
% #region
This part is about making a half-wave rectifier using a single diode. Then to calculate the DC level of the rectified signal. And to display the signal waveform from the oscilloscope.

%================  SINGLE FIGURE  =================================================%
\begin{figure}[h] % Circuit
\centering
\includegraphics[width=\textwidth]{Part6.jpg}
\caption{Half-wave rectifier circuit}
\label{fig:part6}
\end{figure}

\clearpage

%================  SINGLE FIGURE  =================================================%
\begin{figure}[h] % Input signal
\centering
\includegraphics[width=\textwidth]{Part6_InputVoltage.jpg}
\caption{Input AC signal}
\label{fig:part6input}
\end{figure}

\clearpage

%================  SINGLE FIGURE  =================================================%
\begin{figure}[h] % Rectified signal
\centering
\includegraphics[width=\textwidth]{Part6b.jpg}
\caption{Half-wave rectified signal}
\label{fig:Part6b}
\end{figure}

The output from the rectifier is found in Figure~\ref{fig:Part6b} to have a \(V_{peak} = \SI{3,46}{V}\). This means the calculated \(V_{DC} \approx \SI{1,1}{V}\). Filling in this data and the measured DC level using a multimeter we get the following table.

\vspace{1em}

%================  TABLE  =========================================================%
% Table generated by Excel2LaTeX from sheet 'Sheet1'
\begin{table}[htbp] % DC voltages
  \centering
  \caption{Rectified voltage calculated vs measured}
    \begin{tabular}{|l|rl|}
    \hline
    \(V_{peak}\) (Measured) & 3.460 & V \bigstrut\\
    \hline
    \(V_{DC}\) (Calculated) & 1.100 & V \bigstrut\\
    \hline
    \(V_{DC}\) (Measured) & 0.971 & V \bigstrut\\
    \hline
    Difference & 11.750 & \% \bigstrut\\
    \hline
    \end{tabular}%
  \label{tab:part6}%
\end{table}%

\clearpage

%================  SINGLE FIGURE  =================================================%
\begin{figure}[h] % Reversed diode signal
\centering
\includegraphics[width=\textwidth]{Part6e.jpg}
\caption{Reversed diode}
\label{fig:Part6e}
\end{figure}

Reversing the diode gives the following signal waveform as expected.

% #endregion

\end{document}